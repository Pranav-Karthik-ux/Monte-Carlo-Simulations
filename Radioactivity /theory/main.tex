\documentclass[12pt]{article}

% -------------------------------------------------
% PACKAGES
% -------------------------------------------------
\usepackage{amsmath, amssymb}
\usepackage{physics}
\usepackage{graphicx}
\usepackage{bm}
\usepackage{geometry}
\usepackage{hyperref}
\usepackage{mathtools}

\geometry{a4paper, margin=1in}

\title{Monte Carlo Simulation of Radioactivity }
\author{}
\date{}
\begin{document}

\maketitle

\section{Theory of the Monte Carlo Decay Simulation}

In this simulation we model the radioactive decay of an ensemble of $N$ unstable
particles (or nucleons). Radioactive decay is inherently a stochastic process:
each particle has a fixed probability per unit time of decaying, but the exact
moment of decay for each individual particle cannot be predicted. Instead, the
decay follows a probabilistic law.

\subsection{Decay Probability}

For a small time interval $\Delta t$, the probability that a given particle
decays is taken to be a constant value $p$. This corresponds to the discretized
form of an exponential decay law. In the continuous case, the number of
particles $N(t)$ remaining at time $t$ is
\[
N(t) = N_0 e^{-\lambda t},
\]
where $\lambda$ is the decay constant. In the discrete Monte Carlo model, we
approximate the relation between $p$ and $\lambda$ by
\[
p \approx 1 - e^{-\lambda \Delta t}.
\]

\subsection{Monte Carlo Procedure}

At every time step:
\begin{enumerate}
    \item A set of $N(t)$ random numbers, uniformly distributed in $[0,1]$, is
    generated. Each random number corresponds to one particle.
    \item For each particle, if the random number $r_i$ satisfies
    \[
    r_i \le p,
    \]
    then that particle is considered to have decayed during the interval
    $\Delta t$.
    \item The number of undecayed particles $N$ and the number of decayed
    particles $D$ are updated accordingly.
\end{enumerate}

\subsection{Time Evolution}

The process continues while $N(t)>0$ and $t<t_{\max}$. At each step, the
population is updated according to
\[
N(t+\Delta t) = N(t) - \Delta N, \qquad
D(t+\Delta t) = D(t) + \Delta N,
\]
where $\Delta N$ is the number of particles that decayed during the time step.

This algorithm produces a stochastic realization of the decay curve. For large
numbers of particles and small $\Delta t$, the averaged results approach the
analytical exponential decay law.

\subsection{Interpretation}

Because each decay event is determined by random sampling, individual runs of
the simulation will fluctuate around the ideal exponential decay curve. The
Monte Carlo method therefore provides an intuitive and computationally simple
way to visualize and study the probabilistic nature of radioactive decay.
\end{document}
